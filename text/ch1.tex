% Copyright (C) 2018 Liu Yifan, Zhao Tianyu 
% Permission is granted to copy, distribute and/or modify this document
% under the terms of the GNU Free Documentation License, Version 1.3
% or any later version published by the Free Software Foundation;
% with no Invariant Sections, no FrontCover Texts, and no BackCover Texts.
% A copy of the license is included in the section entitled "GNU
% Free Documentation License".
这个模板是大连海事大学数学系本科生毕业论文\LaTeX 模版,中文解决方案是\XeLaTeX 和CTeX。

此版本由数学系2014级应数班Zhao Tianyu和LiuYifan的论文模板改进完成。

并由2016级应数Esther适配Overleaf。
\section{为什么要做这个模板}\label{ch:intr}


因为使用Microsoft 
Word写学术论文是一件非常自我摧残的事情,要花费大量的时间去学习Word排版,
弄不好还要花费别人大量的时间帮忙排自己的论文,还要搭上人情,请人家吃饭什么的,
在别人心中留下一个小小的印记:“切,连排个版都不会!”

关键是用Word写学术论文本来就是一件十分不靠谱的事情,而且完全有更简单,更专业的解决方法 —— {\LaTeX} !

既然你看到这个模板,又能get到用word的痛苦,(Ps:使用Mac,Linux,FreeBSD的同学就不用纠结了,直接投入\LaTeX 的怀抱吧.) 而且有看文档的习惯,那么恭喜你吞下了那颗红色小药丸!也欢迎推荐给周围同学及学弟妹,不限于数学系.

Welcome to the real world!

\subsection{注意事项}
\label{sec:fastguide}
下面一行使用了\textbackslash subsubsection 标题
\subsubsection{注意事项}
\label{sec:features}

\textbf{年代久远,请留意后续学校论文格式变动}


\textbf{重要的事情现在前面,如果无法编译,第一次编译点左上角选 Compiler(编译器) 为 \XeLaTeX,再按ctrl+Enter(右上complie/recompile按钮), (mac用户shortcut需要自己查询)}


建议完整看完本教程,没时间也要看一下面的部分重点和建议,和第四章.
\begin{enumerate}
    \item 如果你通过其他途径得到此模板,又是\LaTeX 初学者,不妨试试在线编译\LaTeX . 在google上搜索"overleaf Templates"(貌似有中文网页站点)进入页面以后搜索"dlmu"或者"大连海事"就可以找到本模板兼容overleaf线上编译的版本.\textbf{重要的事情再说一遍,第一次编译点左上角选 Compiler 为 \XeLaTeX }
    \item 配置在Dmusetup.tex中 
    \item /fig /photo /text 等文件夹即字面意思,cover.pdf是用word模板生成的前两页,main.tex是主文件
    \item 知网查重提交pdf 会乱码 可转成word再进行查重, Linux 下转换word命令行为: 
    \begin{lstlisting}
pandoc ./main.tex --bibliography refs.bib  -o output.doc
#带reference 不推荐
pandoc ./main.tex   -o output.doc 
#推荐不带reference 并需要根据报错删除致谢一类的章节
  \end{lstlisting}
  \item 各位置的题目 设定没有同步,须设定多处,e.g., 页眉
  \item 有不懂的问题,先看overleaf的document\href{https://www.overleaf.com/learn}{"点此"},和stackoverflow自行搜索. 若还有问题无法结局 or 学校论文规范有巨幅更新 or 有意contribution 请到本项目主页\href{https://github.com/XWEster97/DLMU-Bachelor-Thesis-Template}{"点击此处"}开issues(这是联系作者的唯一办法)或pull request.
  \item 引用请分清楚\cite{hart_initializing_2012}和\upcite{hart_initializing_2012} 的区别
  \item cover页生成,本文的即为错误范例,e.g.装订线不应横排,应用Virtual printer输出正确竖排装订线,其他请遵循学校标准.
\end{enumerate}

 





