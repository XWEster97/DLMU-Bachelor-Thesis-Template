% Copyright (C) 2018 Liu Yifan, Zhao Tianyu 
% Permission is granted to copy, distribute and/or modify this document
% under the terms of the GNU Free Documentation License, Version 1.3
% or any later version published by the Free Software Foundation;
% with no Invariant Sections, no FrontCover Texts, and no BackCover Texts.
% A copy of the license is included in the section entitled "GNU
% Free Documentation License".

\section{为什么要做这个模板}\label{ch:intr}


因为使用Microsoft 
Word写学术论文是一件非常自我摧残的事情,要花费大量的时间去学习Word排版,
弄不好还要花费别人大量的时间帮忙排自己的论文,还要搭上人情,请人家吃饭什么的,
在别人心中留下一个小小的印记:“切,连排个版都不会!”

关键是用Word写学术论文本来就是一件十分不靠谱的事情,而且完全有更简单,更专业的解决方法 —— {\LaTeX} !

既然你下载了这个模板,而且有看文档的好习惯,那么恭喜你吞下了那颗红色小药丸!

Welcome to the real world!

其实下面的内容你都没有必要看的,就是例行的啰嗦啰嗦,你可以直接跳到
第\ref{ch:install}章。(看到了吧,pdf文档里面可以有超链接,你Word弄个超链接
出来费劲死了。)

\subsection{模板说明}
\label{sec:fastguide}

\subsubsection{模板特性}
\label{sec:features}

这个模板是大连海事大学数学系本科生毕业论文\LaTeX{}模版,中文解决方案是\XeLaTeX和CTex。

此版本由数学系应数2014级Zhao Tianyu和LiuYifan的论文模板改进完成。

由2016级Esther适配Overleaf。

\textbf{年代久远,请留意后续学校论文格式变动}


\textbf{重要的事情现在前面,第一次编译点左上角选 Compiler 为 \XeLaTeX}
配置在Dmusetup.tex中 

建议完整看完本教程,没时间也要看第四章, 下面是一部分重点和建议
\begin{enumerate}
    \item 配置在Dmusetup.tex中 
    \item /fig /photo /text 等文件夹即字面意思
    \item 知网查重提交pdf 会乱码 可转成word再进行查重, Linux 下转换word命令行为: 
    \begin{lstlisting}
pandoc ./main.tex --bibliography refs.bib  -o output.doc
#带reference 不推荐
pandoc ./main.tex   -o output.doc 
#推荐不带reference 并需要根据报错删除致谢一类的章节
  \end{lstlisting}
  \item 各位置的题目 设定没有同步,须设定多处,e.g., 页眉
  \item 有不懂的问题,先看overleaf的document\href{https://www.overleaf.com/learn}{"点此"},和stackoverflow自行搜索. 若还有问题到本项目主页"点击此处"开issues反馈.
  \item 引用分清楚\cite{hart_initializing_2012}和\upcite{hart_initializing_2012} 的区别
\end{enumerate}

 





