% Copyright (C) 2018 Liu Yifan, Zhibo Xiao and Vital Delmas Mabonzo
% Copyright (C) 2018 Liu Yifan, Zhao Tianyu 
% Permission is granted to copy, distribute and/or modify this document
% under the terms of the GNU Free Documentation License, Version 1.3
% or any later version published by the Free Software Foundation;
% with no Invariant Sections, no FrontCover Texts, and no BackCover Texts.
% A copy of the license is included in the section entitled "GNU
% Free Documentation License".

\section{模板的安装使用}\label{ch:install}

\textbf{使用平台编译请看视频教程,e.g.,\href{https://www.youtube.com/playlist?list=PLnC5h3PY-znyDQKn3knfXfekZLgWyL7QW}{Minimal examples of LaTeX with overleaf (v2)}}.

下面的文字教程适用于本地编译

原本内容已经被注释掉,想看请到源码ch2.tex中看,选中下面文字部分  ctrl+/  解除注释 再编译

% \subsection{安装 \TeX 系统}
% \subsubsection{Windows}
% 考虑到Windows系统用户大多数使用CTex(基于MiKTeX),这里只说明CTex的安装方法。
% \begin{enumerate}
% \item 下载CTex v2.9完整版,\url{http://www.ctex.org/CTeXDownload};
% \item 按照第\ref{sect:compile}节的方法编译你的论文即可。
% \end{enumerate}

% \subsubsection{Mac OSX}

% 完整安装MacTeX套装即可。

% \subsubsection{Linux}

% 完整安装TeXLive最新版即可。




% \subsection{编译你的论文}\label{sect:compile}

% 论文的中英文封面和原创性声明需要使用Word编辑,并且在每页之后添加一个空白页(Ctrl+Enter),然后将前六页保存成pdf文件——cover.pdf。

% 之后在body文件夹用文本编辑器打开各个源文件进行编辑。

% \subsubsection{文本编辑器}

% 不要使用CTex套装中预制了WinEdt编辑器,否则后果自负。
% 由于WinEdt使用GBK编码。插入参考文献时需要配合另外的软件
% 使用,大多数软件默认的编码都是UTF8,所以不推荐WinEdt,我们推荐使用
% TexStudio、Vim+LaTexSuite等支持UTF8编码的专用LaTeX编辑器进行编辑。
% 也可以使用Sublime Text 2、Notepad++等其他纯文本编辑器。

% 文本编辑器推荐列表(依照推荐顺序排序):

% \begin{itemize}
% \item IDE组
% \begin{enumerate}
% 	\item TeXstudio
% 	\item Texmaker
% 	\item TeXnicCenter
% \end{enumerate}
% \item{纯文本编辑器组}
% \begin{enumerate}
% \item Vim+LaTeXSuite (神器)
% \item Sublime Text 2
% \end{enumerate}
% \end{itemize}


% \subsubsection{快捷编译}

% 如果确保使用的\LaTeX 命令没有错误,可以使用模板附带的批处理命令进行编译。
% \subsubsection{Windows下编译}

% 双击 \texttt{run.bat}文件进行编辑。

% \subsubsection{Mac OSX和Linux下编译}

% 在你最喜欢的bash下面运行\texttt{make} 编译。
% 或者运行\texttt{sh run.sh}编译。

% \subsubsection{使用命令行编译}

% 这种方式在所有系统下都可以进行。打开命令行工具,进入模板所在文件夹,按照下面顺序依次输入:
% \begin{verbatim}
% xelatex dmuthesis
% bibtex dmuthesis
% xelatex dmuthesis
% xelatex dmuthesis
% \end{verbatim}
% 为了生成参考文献和书签以及交叉引用,所有的\LaTeX 文件都需要编译4次,\XeLaTeX 
% 文件也不例外。

