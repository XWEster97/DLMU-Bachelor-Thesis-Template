% Copyright (C) 2018 Liu Yifan, Zhao Tianyu 
% Permission is granted to copy, distribute and/or modify this document
% under the terms of the GNU Free Documentation License, Version 1.3
% or any later version published by the Free Software Foundation;
% with no Invariant Sections, no FrontCover Texts, and no BackCover Texts.
% A copy of the license is included in the section entitled "GNU
% Free Documentation License".
\usepackage{ctex}
\setCJKmainfont[AutoFakeBold=2, ItalicFont={simsun.ttc}]{simsun.ttc}
\setmainfont{Times New Roman} 
\usepackage{fontspec}
\usepackage{amsmath}
\usepackage{amssymb}
\usepackage{hyperref} 
\usepackage{amsthm} 
\usepackage{mathtools} 
\usepackage{mathrsfs}
\usepackage{fancyhdr}
\usepackage{enumerate}
\usepackage{metalogo}
% \usepackage{mathscr}
\numberwithin{equation}{section}
\newtheorem{definition}{定义}[section]
\newtheorem{example}{例}[section]
\newtheorem{theorem}{定理}[section]
\newtheorem*{maintheorem}{主要定理}
\newtheorem{question}{问题}
\newtheorem{lemma}[theorem]{引理}
\newtheorem{remark}[theorem]{备注}
\newtheorem{corollary}[theorem]{推论}
\newtheorem{proposition}[theorem]{命题}
%\songti
% \newtheorem{algorithm}
\ctexset{section={format={\zihao{3}\bfseries},
			name = {第,章},beforeskip=0.5ex plus 20pt,afterskip=10pt}}
\ctexset{subsection={format={\zihao{4}\bfseries },
				beforeskip=0.75ex,afterskip=0.95ex}}
\ctexset{subsubsection={format={\zihao{-4}\bfseries },
				beforeskip=0.75ex,afterskip=0.95ex}}

\usepackage{caption}
\captionsetup{font={small},labelsep=space}
\numberwithin{figure}{section}
\renewcommand{\proofname}{\zihao{-4}  \bfseries 证明}
\usepackage{geometry}
\geometry{top=2.5cm,bottom=2.5cm,left=3cm,right=2.5cm,headsep=0.5cm}
%\setmainfont{Times New Roman}
\ctexset{abstractname={\zihao{3}   摘\hspace*{2em} 要}}
%\ctexset{abstract={beforskip={20pt}
		%	     afterskip=10pt}}
\ctexset{bibname={\centerline{ \zihao{3}  参\hspace*{0.5em}考\hspace*{0.5em}文\hspace*{0.5em}献}}}
%\ctexset{section ={name = {第,章},beforeskip=20pt,afterskip=10pt}}
\usepackage{setspace}

\usepackage{listings} 
\usepackage{natbib}
\setcitestyle{numbers,square,comma}
%\setlength{\bibspacing}{\baselineskip}

\newcommand{\upcite}[1]{\textsuperscript{\cite{#1}}}

\usepackage{titletoc}
\titlecontents{section}[0pt]{\addvspace{2pt}\filright}
{\contentspush{\thecontentslabel\ }}
{}{\titlerule*[8pt]{.}\contentspage}
%%%%commented%%%%
%\usepackage{enumitem}
%\setenumerate{fullwidth,itemindent=\parindent,listparindent=\parindent,itemsep=0ex,partopsep=0pt,parsep=0ex}

\usepackage{graphicx}
\usepackage{caption}
\usepackage{subfigure}
%%%%%%%%%%%%%5
\usepackage{enumerate}
\usepackage{minted}
%%%%%%%%%%%%%
\usepackage{pdfpages}
\usepackage{afterpage}
\allowdisplaybreaks[1]


\newcommand{\h}{\tilde{h}}
\renewcommand{\k}{\kappa}
\renewcommand{\b}{\beta}
\renewcommand{\t}{\theta}
\newcommand{\la}{\lambda}
\newcommand{\La}{\Lambda}
\newcommand{\kk}{\tilde{\kappa}}
\renewcommand{\L}{\mathscr{L}}
\newcommand{\A}{\mathscr{A}}
\newcommand{\F}{\mathbf{F}}
\newcommand{\p}{\mathbf{p}}
%%%%%%%%%%%%%%%%%fanfan%%%%%%

\newcommand{\blank}[1]{\hspace*{#1}}
\newcommand{\sectionend}{\ifodd\value{page}\afterpage{\null\newpage}\else\fi\newpage\vspace*{0pt}}
\usepackage{dutchcal}
%%%%%%%%%%%%%%%%%%%%%%%%%%%%%%%

\pagestyle{fancy}
\lhead{}
\rhead{}
\chead{\zihao{-5} 论文题目}
\cfoot{\zihao{-5} \thepage}
\renewcommand{\headrulewidth}{0.4pt}
\renewcommand{\footrulewidth}{0pt}



\title{\zihao{2} \bfseries  论文题目}
\author{}
\date{}



% 表格中支持跨行
%\usepackage{multirow}

% 跨页表格
%\usepackage{longtable}

% 固定宽度的表格
%\usepackage{tabularx}

% 表格中的反斜线
%\usepackage{diagbox}

% 确定浮动对象的位置,可以使用 H,强制将浮动对象放到这里(可能效果很差)
%\usepackage{float}

% 浮动图形控制宏包。
% 允许上一个 section 的浮动图形出现在下一个 section 的开始部分
% 该宏包提供处理浮动对象的 \FloatBarrier 命令,使所有未处
% 理的浮动图形立即被处理。这三个宏包仅供参考,未必使用:
% \usepackage[below]{placeins}
% \usepackage{floatflt} % 图文混排用宏包
% \usepackage{rotating} % 图形和表格的控制旋转

% 定理类环境宏包
%\usepackage[amsmath,thmmarks,hyperref]{ntheorem}


% 数学命令
%\input{math_commands.tex}

% 定义自己常用的东西
% \def\myname{薛瑞尼}




% hyperref 宏包在最后调用
\usepackage{hyperref}
% algorithms
\usepackage[ruled,vlined]{algorithm2e}
\renewcommand{\algorithmcfname}{算法}

\usepackage{booktabs}
% \usepackage{xcolor}
% 据说加了一行这个查重就不会乱码了
% \usepackage{ccmap}